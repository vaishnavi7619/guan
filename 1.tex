\let\negmedspace\undefined
\let\negthickspace\undefined
\documentclass[journal,12pt,twocolumn]{IEEEtran}
\usepackage{cite}
\usepackage{amsmath,amssymb,amsfonts,amsthm}
\usepackage{algorithmic}
\usepackage{graphicx}
\usepackage{textcomp}
\usepackage{xcolor}
\usepackage{txfonts}
\usepackage{listings}
\usepackage{enumitem}
\usepackage{mathtools}
\usepackage{gensymb}
\usepackage{comment}
\usepackage[breaklinks=true]{hyperref}
\usepackage{tkz-euclide} 
\usepackage{listings}                                     
\def\inputGnumericTable{}                                 
\usepackage[latin1]{inputenc}                                
\usepackage{color}                                            
\usepackage{array}                                            
\usepackage{longtable}                                       
\usepackage{calc}                                             
\usepackage{multirow}                                         
\usepackage{hhline}                                           
\usepackage{ifthen}                                           
\usepackage{lscape}

\newtheorem{theorem}{Theorem}[section]
\newtheorem{problem}{Problem}
\newtheorem{proposition}{Proposition}[section]
\newtheorem{lemma}{Lemma}[section]
\newtheorem{corollary}[theorem]{Corollary}
\newtheorem{example}{Example}[section]
\newtheorem{definition}[problem]{Definition}
\newcommand{\BEQA}{\begin{eqnarray}}
\newcommand{\EEQA}{\end{eqnarray}}
\newcommand{\define}{\stackrel{\triangle}{=}}
\theoremstyle{remark}
\newtheorem{rem}{Remark}
\begin{document}

\bibliographystyle{IEEEtran}
\vspace{3cm}

\title{NCERT 12.8 Q4}
\author{EE23BTECH11014 - Devarakonda Guna Vaishnavi $^{}$% <-this % stops a space
}
\maketitle
\newpage
\bigskip

\renewcommand{\thefigure}{\theenumi}
\renewcommand{\thetable}{\theenumi}

\bibliographystyle{IEEEtran}


\textbf{Question:} A plane electromagnetic wave travels in vacuum along the \(z\)-direction. What can you say about the directions of its electric (\(\mathbf{E}\)) and magnetic (\(\mathbf{B}\)) field vectors? If the frequency of the wave is \(30 \, \text{MHz}\), what can you say about its wavelength?
 

\textbf{Solution:} 

\textbf{Method } 
A plane electromagnetic wave travels in vacuum along the \(z\)-direction. The electric (\(\mathbf{E}\)) and magnetic (\(= 30 \times 10^6 \, \text{Hz}\), you can plug in the values to find the wavelength:
\[ \lambda = \frac{3 \times 10^8 \, \text{m/s}}{30 \times 10^6 \, \text{Hz}} \]

This will give you the wavelength of the electromagnetic wave.
\mathbf{B}\)) field vectors are perpendicular to each other and both are perpendicular to the direction of propagation. Specifically, if the wave is traveling along the \(z\)-direction, then \(\mathbf{E}\) and \(\mathbf{B}\) will be in the \(x\)- and \(y\)-directions.

The relationship between frequency (\(f\)), wavelength (\(\lambda\)), and the speed of light (\(c\)) is given by the formula:
\[ c = f \lambda \]

where:
\(c\) is the speed of light in a vacuum (\(3 \times 10^8 \, \text{m/s}\)),
\(f\) is the frequency of the wave, and
\(\lambda\) is the wavelength.

You can rearrange this formula to solve for the wavelength:
\[ \lambda = \frac{c}{f} \]

Given a frequency of \(30 \, \text{MHz} % \iffalse
\(\lambda = 10 \, \text{m}\)






\end{document}
